\chapter{AI in the Energy Sector}
\label{cha:ai_in_the_energy_sector}

\section{Introduction}

The energy sector is undergoing a profound transformation, driven by the dual imperatives of decarbonisation and decentralisation. Artificial intelligence (AI) is emerging as a critical enabling technology in this transition, offering powerful tools to enhance efficiency, improve reliability, and integrate renewable energy sources into the grid. This chapter explores the diverse applications of AI across the energy value chain, from optimising power generation and transmission to enabling smarter energy consumption. It also examines the challenges and opportunities associated with the deployment of AI in this critical sector \parencite{zepter2019review}.

\section{Key Applications of AI in the Energy Sector}

\subsection{Grid Management and Optimisation}

AI is playing a pivotal role in the development of smart grids, which are essential for managing the complexity of modern power systems. AI algorithms can analyse real-time data from sensors and smart meters to forecast electricity demand, optimise power flows, and prevent blackouts. By enabling a more dynamic and responsive grid, AI can improve grid stability, reduce energy losses, and facilitate the integration of intermittent renewable energy sources such as solar and wind \parencite{eera2025artificial}.

\subsection{Predictive Maintenance}

In the energy sector, equipment failures can have significant economic and safety consequences. AI-powered predictive maintenance systems can help to prevent these failures by analysing data from sensors to detect anomalies and predict when equipment is likely to fail. This allows energy companies to shift from a reactive to a proactive maintenance approach, reducing downtime, lowering maintenance costs, and improving the safety and reliability of energy infrastructure \parencite{aliyu2022review}.

\subsection{Renewable Energy Forecasting}

The intermittency of renewable energy sources like wind and solar poses a significant challenge for grid operators. AI is helping to address this challenge by providing more accurate forecasts of renewable energy generation. By analysing weather data, satellite imagery, and historical generation data, AI models can predict the output of wind and solar farms with increasing accuracy. These improved forecasts enable grid operators to better manage the variability of renewable energy and ensure a reliable supply of electricity \parencite{aliyu2022review}.

\subsection{Energy Efficiency and Consumption}

AI can also be used to improve energy efficiency and optimise energy consumption in buildings, industry, and transport. AI-powered systems can analyse energy consumption patterns, identify opportunities for energy savings, and automatically adjust energy usage to reduce waste. For example, smart thermostats can learn a household's preferences and automatically adjust heating and cooling to save energy, while AI-powered systems in factories can optimise industrial processes to reduce energy consumption \parencite{zepter2019review}.

\section{Challenges and the Future}

Despite the enormous potential of AI in the energy sector, there are several challenges that need to be addressed. These include the need for large amounts of high-quality data, the risk of cyberattacks on AI-enabled energy systems, and the ethical implications of using AI to make decisions that affect people's access to energy. Overcoming these challenges will require a concerted effort from policymakers, regulators, and the energy industry to develop appropriate standards, regulations, and best practices for the responsible use of AI \parencite{eera2025artificial}.

\section{Conclusion}

Artificial intelligence is set to play a transformative role in the energy sector, helping to create a more sustainable, reliable, and affordable energy future. From optimising the grid to empowering consumers, the applications of AI are vast and varied. However, realising the full potential of AI in the energy sector will require a proactive and collaborative approach to addressing the associated challenges. By embracing innovation while ensuring safety, security, and equity, we can harness the power of AI to accelerate the transition to a clean energy future.