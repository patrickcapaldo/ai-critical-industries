\chapter{AI in Transport}
\label{cha:ai_in_transport}

\section{Introduction}

The transport sector is undergoing a period of unprecedented change, driven by the convergence of several megatrends, including urbanisation, decarbonisation, and digitalisation. Artificial intelligence (AI) is a key enabling technology that is accelerating this transformation, with the potential to create a transport system that is safer, more efficient, and more sustainable. This chapter provides an overview of the key applications of AI in transport, from autonomous vehicles and intelligent traffic management to personalised mobility services and predictive maintenance \parencite{ma2020review}.

\section{Key Applications of AI in Transport}

\subsection{Traffic Management and Optimisation}

Traffic congestion is a major problem in many cities around the world, leading to wasted time, increased pollution, and economic losses. AI is being used to develop intelligent traffic management systems that can optimise traffic flow and reduce congestion. These systems use real-time data from a variety of sources, including traffic sensors, GPS devices, and social media, to predict traffic conditions and dynamically adjust traffic signals, ramp meters, and speed limits. By optimising traffic flow, these systems can reduce travel times, improve air quality, and enhance the overall efficiency of the transport network \parencite{abduljabbar2019applications}.

\subsection{Public Transportation}

AI is also being used to improve the efficiency and attractiveness of public transport. AI-powered systems can be used to optimise bus and train schedules, predict passenger demand, and provide real-time information to passengers. For example, AI can be used to develop on-demand bus services that can be booked through a mobile app, providing a more convenient and flexible alternative to traditional fixed-route services. By making public transport more efficient and user-friendly, AI can help to reduce our reliance on private cars and create more sustainable cities \parencite{isalkar2024artificial}.

\subsection{Autonomous Vehicles}

The development of autonomous vehicles is one of the most exciting and transformative applications of AI in transport. Self-driving cars, trucks, and buses have the potential to revolutionise the way we travel, making our roads safer, reducing congestion, and providing new mobility options for the elderly and disabled. AI is the core technology that enables autonomous vehicles to perceive their environment, make decisions, and control their movements. While fully autonomous vehicles are still some years away from widespread deployment, the technology is developing rapidly and is already being used in a variety of applications, such as automated parking and driver assistance systems \parencite{ma2020review}.

\subsection{Predictive Maintenance}

AI is also being used to improve the maintenance of transport infrastructure and vehicles. AI-powered predictive maintenance systems can analyse data from sensors to detect early signs of wear and tear, and predict when maintenance is likely to be required. This allows transport operators to schedule maintenance proactively, before a failure occurs, which can help to reduce downtime, lower maintenance costs, and improve safety \parencite{abduljabbar2019applications}.

\subsection{Safety and Security}

AI has the potential to significantly improve safety and security in the transport sector. AI-powered driver assistance systems can help to prevent accidents by warning drivers of potential hazards and even taking control of the vehicle in an emergency. AI can also be used to improve security at airports, ports, and border crossings, by automatically detecting suspicious behaviour and identifying potential threats \parencite{isalkar2024artificial}.

\section{Conclusion}

Artificial intelligence is set to have a profound impact on the transport sector, with the potential to create a more efficient, sustainable, and user-friendly transport system. However, the deployment of AI in transport also raises a number of challenges, including the need for new regulations, the ethical implications of autonomous decision-making, and the potential for job losses. Addressing these challenges will require a collaborative effort from policymakers, industry, and academia to ensure that AI is deployed in a way that benefits society as a whole.